%!TEX TS-program = xelatex
\documentclass[11pt]{article}

\usepackage[english]{babel}

\usepackage{amsmath,amssymb,amsfonts}
\usepackage[utf8]{inputenc}
\usepackage[T1]{fontenc}
\usepackage{stix2}
\usepackage[scaled]{helvet}
\usepackage[scaled]{inconsolata}

\usepackage{lastpage}

\usepackage{setspace}

\usepackage{ccicons}

\usepackage[hang,flushmargin]{footmisc}

\usepackage{geometry}

\setlength{\parindent}{0pt}
\setlength{\parskip}{6pt plus 2pt minus 1pt}

\usepackage{fancyhdr}
\renewcommand{\headrulewidth}{0pt}\providecommand{\tightlist}{%
  \setlength{\itemsep}{0pt}\setlength{\parskip}{0pt}}

\makeatletter
\newcounter{tableno}
\newenvironment{tablenos:no-prefix-table-caption}{
  \caption@ifcompatibility{}{
    \let\oldthetable\thetable
    \let\oldtheHtable\theHtable
    \renewcommand{\thetable}{tableno:\thetableno}
    \renewcommand{\theHtable}{tableno:\thetableno}
    \stepcounter{tableno}
    \captionsetup{labelformat=empty}
  }
}{
  \caption@ifcompatibility{}{
    \captionsetup{labelformat=default}
    \let\thetable\oldthetable
    \let\theHtable\oldtheHtable
    \addtocounter{table}{-1}
  }
}
\makeatother

\usepackage{array}
\newcommand{\PreserveBackslash}[1]{\let\temp=\\#1\let\\=\temp}
\let\PBS=\PreserveBackslash

\usepackage[breaklinks=true]{hyperref}
\hypersetup{colorlinks,%
citecolor=blue,%
filecolor=blue,%
linkcolor=blue,%
urlcolor=blue}
\usepackage{url}

\usepackage{caption}
\setcounter{secnumdepth}{0}
\usepackage{cleveref}

\usepackage{graphicx}
\makeatletter
\def\maxwidth{\ifdim\Gin@nat@width>\linewidth\linewidth
\else\Gin@nat@width\fi}
\makeatother
\let\Oldincludegraphics\includegraphics
\renewcommand{\includegraphics}[1]{\Oldincludegraphics[width=\maxwidth]{#1}}

\usepackage{longtable}
\usepackage{booktabs}

\usepackage{color}
\usepackage{fancyvrb}
\newcommand{\VerbBar}{|}
\newcommand{\VERB}{\Verb[commandchars=\\\{\}]}
\DefineVerbatimEnvironment{Highlighting}{Verbatim}{commandchars=\\\{\}}
% Add ',fontsize=\small' for more characters per line
\usepackage{framed}
\definecolor{shadecolor}{RGB}{248,248,248}
\newenvironment{Shaded}{\begin{snugshade}}{\end{snugshade}}
\newcommand{\KeywordTok}[1]{\textcolor[rgb]{0.13,0.29,0.53}{\textbf{#1}}}
\newcommand{\DataTypeTok}[1]{\textcolor[rgb]{0.13,0.29,0.53}{#1}}
\newcommand{\DecValTok}[1]{\textcolor[rgb]{0.00,0.00,0.81}{#1}}
\newcommand{\BaseNTok}[1]{\textcolor[rgb]{0.00,0.00,0.81}{#1}}
\newcommand{\FloatTok}[1]{\textcolor[rgb]{0.00,0.00,0.81}{#1}}
\newcommand{\ConstantTok}[1]{\textcolor[rgb]{0.00,0.00,0.00}{#1}}
\newcommand{\CharTok}[1]{\textcolor[rgb]{0.31,0.60,0.02}{#1}}
\newcommand{\SpecialCharTok}[1]{\textcolor[rgb]{0.00,0.00,0.00}{#1}}
\newcommand{\StringTok}[1]{\textcolor[rgb]{0.31,0.60,0.02}{#1}}
\newcommand{\VerbatimStringTok}[1]{\textcolor[rgb]{0.31,0.60,0.02}{#1}}
\newcommand{\SpecialStringTok}[1]{\textcolor[rgb]{0.31,0.60,0.02}{#1}}
\newcommand{\ImportTok}[1]{#1}
\newcommand{\CommentTok}[1]{\textcolor[rgb]{0.56,0.35,0.01}{\textit{#1}}}
\newcommand{\DocumentationTok}[1]{\textcolor[rgb]{0.56,0.35,0.01}{\textbf{\textit{#1}}}}
\newcommand{\AnnotationTok}[1]{\textcolor[rgb]{0.56,0.35,0.01}{\textbf{\textit{#1}}}}
\newcommand{\CommentVarTok}[1]{\textcolor[rgb]{0.56,0.35,0.01}{\textbf{\textit{#1}}}}
\newcommand{\OtherTok}[1]{\textcolor[rgb]{0.56,0.35,0.01}{#1}}
\newcommand{\FunctionTok}[1]{\textcolor[rgb]{0.00,0.00,0.00}{#1}}
\newcommand{\VariableTok}[1]{\textcolor[rgb]{0.00,0.00,0.00}{#1}}
\newcommand{\ControlFlowTok}[1]{\textcolor[rgb]{0.13,0.29,0.53}{\textbf{#1}}}
\newcommand{\OperatorTok}[1]{\textcolor[rgb]{0.81,0.36,0.00}{\textbf{#1}}}
\newcommand{\BuiltInTok}[1]{#1}
\newcommand{\ExtensionTok}[1]{#1}
\newcommand{\PreprocessorTok}[1]{\textcolor[rgb]{0.56,0.35,0.01}{\textit{#1}}}
\newcommand{\AttributeTok}[1]{\textcolor[rgb]{0.77,0.63,0.00}{#1}}
\newcommand{\RegionMarkerTok}[1]{#1}
\newcommand{\InformationTok}[1]{\textcolor[rgb]{0.56,0.35,0.01}{\textbf{\textit{#1}}}}
\newcommand{\WarningTok}[1]{\textcolor[rgb]{0.56,0.35,0.01}{\textbf{\textit{#1}}}}
\newcommand{\AlertTok}[1]{\textcolor[rgb]{0.94,0.16,0.16}{#1}}
\newcommand{\ErrorTok}[1]{\textcolor[rgb]{0.64,0.00,0.00}{\textbf{#1}}}
\newcommand{\NormalTok}[1]{#1}

\newlength{\cslhangindent}
\setlength{\cslhangindent}{1.5em}
\newlength{\csllabelwidth}
\setlength{\csllabelwidth}{3em}
\newenvironment{CSLReferences}[3] % #1 hanging-ident, #2 entry spacing
 {% don't indent paragraphs
  \setlength{\parindent}{0pt}
  % turn on hanging indent if param 1 is 1
  \ifodd #1 \everypar{\setlength{\hangindent}{\cslhangindent}}\ignorespaces\fi
  % set entry spacing
  \ifnum #2 > 0
  \setlength{\parskip}{#2\baselineskip}
  \fi
 }%
 {}
\usepackage{calc} % for \widthof, \maxof
\newcommand{\CSLBlock}[1]{#1\hfill\break}
\newcommand{\CSLLeftMargin}[1]{\parbox[t]{\maxof{\widthof{#1}}{\csllabelwidth}}{#1}}
\newcommand{\CSLRightInline}[1]{\parbox[t]{\linewidth}{#1}}
\newcommand{\CSLIndent}[1]{\hspace{\cslhangindent}#1}\geometry{verbose,letterpaper,tmargin=2.2cm,bmargin=2.2cm,lmargin=2.2cm,rmargin=2.2cm}

\usepackage{lineno}
\usepackage[nolists,noheads]{endfloat}

\pagestyle{plain}

\tolerance=1
\emergencystretch=\maxdimen
\hyphenpenalty=10000
\hbadness=10000

\doublespacing

\fancypagestyle{normal}
{
  \fancyhf{}
  \fancyfoot[R]{\footnotesize\sffamily\thepage\ of \pageref*{LastPage}}
}
\begin{document}
\raggedright
\thispagestyle{empty}
{\Large\bfseries\sffamily The biological interpretation of probabilistic
food webs}
\vskip 5em

%
\href{https://orcid.org/0000-0001-9051-0597}{Francis\,Banville}%
%
\,\textsuperscript{1,2,3,‡}\quad %
\href{https://orcid.org/0000-0001-6067-1349}{Tanya\,Strydom}%
%
\,\textsuperscript{1,3,‡}\quad %
\href{https://orcid.org/0000-0002-0735-5184}{Timothée\,Poisot}%
%
\,\textsuperscript{1,3}

\textsuperscript{1}\,Université de
Montréal\quad \textsuperscript{2}\,Université de
Sherbrooke\quad \textsuperscript{3}\,Quebec Centre for Biodiversity
Science

\textsuperscript{‡}\,These authors contributed equally to the work\\

\textbf{Correspondance to:}\\
Francis Banville --- \texttt{francis.banville@umontreal.ca}\\

\vfill
This work is released by its authors under a CC-BY 4.0 license\hfill\ccby\\
Last revision: \emph{\today}

\clearpage
\thispagestyle{empty}

\vfill


        {\bfseries 1:}\,Community ecologists are increasingly shifting
from a binary thinking of food webs and other ecological networks (e.g.,
do species interact?) to a more probabilistic perspective (e.g., how
likely are species to interact?). Assuredly, the benefits of
representing ecological interactions as probabilistic events are
numerous, from a better assessment of the spatial variation of trophic
interactions to an increase capacity to reconstruct networks from sparse
data.\\%
        {\bfseries 2:}\,However, probabilities need to be used with
caution when working with species interactions. Indeed, depending on the
system at hand and the method used to build probabilistic networks,
probabilities can have different interpretations that imply different
ways to manipulate them. This is rarely discussed in the literature,
thus impeding our ability to use data on probabilistic interactions
appropriately.\\%
        {\bfseries 3:}\,At the core of these differences lie the
distinction between assessing the likelihood that two groups of
individuals \emph{can} interact and the likelihood that they \emph{will}
interact. This impacts the spatial, temporal, and taxonomic scaling of
interaction probabilities, thus further enlightening the need to
properly define them in their ecological context.\\%
        {\bfseries 4:}\,With these challenges in mind, we propose a
general approach to thinking about probabilities in regards to
ecological interactions, with a strong focus on food webs, and call for
better definitions and conceptualizations of probabilistic ecological
networks, both at the local and regional scales.\\%
    
\vfill

\clearpage
\linenumbers
\pagestyle{normal}

\hypertarget{introduction}{%
\section{Introduction}\label{introduction}}

\hypertarget{general-background}{%
\subsection{General background}\label{general-background}}

Why is it useful to think about interactions as probabilistic events?

\begin{itemize}
\tightlist
\item
  A biological interaction is probabilistic since two taxa co-occurring
  does not mean they are going to meet (e.g., think of species relative
  abundances). Also, two individuals meeting does not mean that an
  interaction will occur (e.g., a lion crossing paths with a gazelle
  does not mean predation).
\item
  Species interactions are contextual on the environment and on the
  physiological state of both species (or individuals).
\item
  Representing trophic interactions as probabilistic events helps us
  predict food webs across time and space and assess their spatial
  variability.
\end{itemize}

\textbf{Papers:} Poisot \emph{et al.} (2016)

\hypertarget{problem-and-objectives}{%
\subsection{Problem and objectives}\label{problem-and-objectives}}

Why should we use probabilities with caution when working with food webs
and other ecological networks? What is the objective of this paper?

\begin{itemize}
\tightlist
\item
  There are different ways to define and interpret interaction
  probabilities depending on the study system and on the method used to
  build probabilistic food webs.
\item
  It is important to document and define what we mean by an interaction
  probability because different definitions can have different
  ecological and statistical implications and interpretation.
\item
  This paper aims to outline some of the challenges in interpreting
  interaction probabilities in food webs and propose an approach to
  thinking about them. Using clear and simple mathematical equations, we
  distinguish different meanings of probabilistic interactions.
\end{itemize}

\hypertarget{definitions-and-interpretations}{%
\section{Definitions and
interpretations}\label{definitions-and-interpretations}}

\hypertarget{overview-of-interaction-probabilities}{%
\subsection{Overview of interaction
probabilities}\label{overview-of-interaction-probabilities}}

How are interaction probabilities defined in the literature? It might
not be as intuitive as one would think.

\begin{itemize}
\tightlist
\item
  There is a big difference in how we interpret the probability that two
  species \emph{can} interact (metaweb) and the probability that they
  \emph{will} interact (realized networks).
\item
  Interaction probabilities can be used to describe Boolean interactions
  (e.g., the probability that two species interact) and weighted
  interactions (e.g., the probability distribution of the amount of
  energy flow between two species).
\item
  In many studies, it is not obvious if authors use interaction scores
  or probabilities (in the latter case, it is rarely specified what
  these probabilities truly represent).
\end{itemize}

\hypertarget{probabilistic-metawebs}{%
\subsection{Probabilistic metawebs}\label{probabilistic-metawebs}}

What does a probability in the context of a metaweb mean?

\begin{itemize}
\tightlist
\item
  It means the probability that two taxa can interact, regardless of
  biological plasticity, environmental variability, or co-occurrence.
\item
  One observation is enough to set this probability to one.
\item
  Can we turn this into a local network realisation that is also
  probabilistic and intuitive?
\end{itemize}

\textbf{Papers:} Strydom \emph{et al.} (2022)

\hypertarget{probabilistic-local-networks}{%
\subsection{Probabilistic local
networks}\label{probabilistic-local-networks}}

What does a probability in the context of a local network mean? A
cautionary tale of how we define probabilities.

\begin{itemize}
\tightlist
\item
  It means the probability that two taxa will interact at a given
  location.
\item
  What do we mean by saying that two taxa will interact? We usually mean
  that at least one individual from one group will interact with (e.g.,
  predate) at least one other individual from the other group.
\item
  The probability is conditional on the environmental and local
  abundance contexts.
\item
  We should expect a certain number of interactions to be realized
  depending on the probability value. This number depends on the number
  of trials, which also depends on the ecological context (e.g.,
  environmental conditions, scale) in which probabilities were
  estimated. This is in contrast with probabilities in metawebs.
\end{itemize}

\hypertarget{scaling}{%
\section{Scaling}\label{scaling}}

\hypertarget{spatial-and-temporal-scales}{%
\subsection{Spatial and temporal
scales}\label{spatial-and-temporal-scales}}

How do interaction probabilities scale spatially and temporally?

\begin{itemize}
\tightlist
\item
  Why do probabilistic local food webs scale with area and time but not
  probabilistic metawebs?
\item
  In metawebs, interaction probabilities do not scale with space and
  time because they depend solely on the biological capacity of two
  species to interact.
\item
  In local food webs, interaction probabilities scale with space and
  time because there are more opportunities of interactions (e.g., more
  environmental conditions) in a larger area and longer time period.
\item
  What are some network area relationships in probabilistic local food
  webs?
\item
  We know that local networks can inform regional networks. However, can
  regional networks inform local networks?
\end{itemize}

\textbf{Figure:} Empirical example of the association between the number
of interactions in realized local food webs and the number of
interactions in the corresponding species subnetworks of regional
networks. We should expect the interaction to be linear below the 1:1
line.

\textbf{Papers:} there might be something in these McLeod \emph{et al.}
(2020); McLeod \emph{et al.} (2021); Wood \emph{et al.} (2015)

\hypertarget{taxonomic-scale}{%
\subsection{Taxonomic scale}\label{taxonomic-scale}}

How do interaction probabilities scale taxonomically?

\begin{itemize}
\tightlist
\item
  There are different biological interpretations of probabilities in
  food webs at the individual level and at higher taxonomic levels.
\item
  How does the scaling up of the nodes from an individual to population
  to any higher taxonomic group change our interpretation of interaction
  probabilities? How does the aggregation change our interpretation?
\item
  How is it similar and different to spatial and temporal scaling?
  Basically, all kinds of scaling are just different ways to aggregate
  individuals or nodes.
\end{itemize}

\textbf{Figure:} Conceptual figure of how a scale up of the nodes from
an individual to a population to any higher taxonomic group change our
interpretation of the probability of interaction.

\hypertarget{concluding-remarks}{%
\section{Concluding remarks}\label{concluding-remarks}}

Here we present some advice moving forward.

\begin{itemize}
\tightlist
\item
  What can we learn from other systems/fields (e.g., social networks,
  probabilistic graph theory)?
\item
  What even are probabilities? What is the probability that we will ever
  know the answer to that?
\item
  Be careful of how we define probabilities. Be sure to be explicit
  about these things. Be sure to specify the type of interaction, the
  spatial, temporal, and taxonomic scale when presenting new data on
  interaction probabilities. We need better metadata documentation.
\item
  Be careful to use and manipulate interaction probabilities properly
  depending on how they were defined and obtained. Different
  interpretations imply different scaling, and thus different ways to
  manipulate these numbers.
\item
  Maybe mention thinking about a workflow to predict probabilistic local
  food webs from probabilistic metawebs.
\end{itemize}

\hypertarget{references}{%
\section*{References}\label{references}}
\addcontentsline{toc}{section}{References}

\hypertarget{refs}{}
\begin{CSLReferences}{1}{0}
\leavevmode\hypertarget{ref-McLeod2021SamAsy}{}%
McLeod, A., Leroux, S.J., Gravel, D., Chu, C., Cirtwill, A.R., Fortin,
M.-J., \emph{et al.} (2021). Sampling and asymptotic network properties
of spatial multi-trophic networks. \emph{Oikos}, n/a.

\leavevmode\hypertarget{ref-McLeod2020EffSpe}{}%
McLeod, A.M., Leroux, S.J. \& Chu, C. (2020). Effects of species traits,
motif profiles, and environment on spatial variation in multi-trophic
antagonistic networks. \emph{Ecosphere}, 11, e03018.

\leavevmode\hypertarget{ref-Poisot2016StrProa}{}%
Poisot, T., Cirtwill, A.R., Cazelles, K., Gravel, D., Fortin, M.-J. \&
Stouffer, D.B. (2016). The structure of probabilistic networks.
\emph{Methods in Ecology and Evolution}, 7, 303--312.

\leavevmode\hypertarget{ref-Strydom2022FooWeb}{}%
Strydom, T., Bouskila, S., Banville, F., Barros, C., Caron, D., Farrell,
M.J., \emph{et al.} (2022). Food web reconstruction through phylogenetic
transfer of low-rank network representation. \emph{Methods in Ecology
and Evolution}, n/a.

\leavevmode\hypertarget{ref-Wood2015EffSpa}{}%
Wood, S.A., Russell, R., Hanson, D., Williams, R.J. \& Dunne, J.A.
(2015). Effects of spatial scale of sampling on food web structure.
\emph{Ecology and Evolution}, 5, 3769--3782.

\end{CSLReferences}

\end{document}
